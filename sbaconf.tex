% !TEX root = sbaconf.tex
%===============================================================================
% $Id: ifacconf.tex 19 2011-10-27 09:32:13Z jpuente $  
% Template for IFAC meeting papers
% Copyright (c) 2007-2008 International Federation of Automatic Control
%===============================================================================
\documentclass[a4paper]{ifacconf}

\usepackage{graphicx,amsmath,url}      % include this line if your document contains figures
\usepackage[round]{natbib}             % required for bibliography
%===============================================================================

% If in Portuguese or Spanish, choose
\def\portugues{1} 
\usepackage[spanish,brazil,english]{babel}
\usepackage[T1]{fontenc}
\usepackage[utf8]{inputenc}
\usepackage{unicode}
\usepackage{ae}
\usepackage{placeins}

\if\portugues1
% =====================================================================
% =====================================================================
% If the manuscript is in Spanish, please change the texts adequatelly.
% You may also add other definitions in this part.
 \newtheorem{teorema}[thm]{{\em Teorema}}{ }
 \newtheorem{lema}[thm]{{\em Lema}}{ }
 \newtheorem{corolario}[thm]{{\em Corolário}}{ }
 \newenvironment{prova}{{\bf Prova.}}{ }
% ===============================================================
\fi

\begin{document}
	
    \selectlanguage{brazil}
	
    \begin{frontmatter}
        
        \title{Trabalho computacional \\Teoria de Sistemas Lineares \\Síntese de observador e controlador por realimentação de estados para controle de suspensão veicular ativa, não-linear, através de amortecedor magneto-reológico como atuador} 

        \author[First]{Charles Quirino Pimenta} 
        
        \address[First]{Programa de Pós-Graduação em Engenharia Elétrica - Universidade Federal de 
Minas Gerais - Av. Antônio Carlos 6627, 31270-901, Belo Horizonte, MG, Brasil\\ e-mail:charlesqp@ufmg.br.}
        
        \selectlanguage{english}
        \renewcommand{\abstractname}{{\bf Abstract:~}}
        \begin{abstract} Colocar o abstract em ingles aqui
        
        \vskip 1mm% não altere esse espaçamento
        \selectlanguage{brazil}
        {\noindent \bf Resumo}:Colocar o abstract em português aqui
        \end{abstract}
        
        \selectlanguage{english}
        
        \begin{keyword} Colocar palavres chave em ingles aqui 
        
        \vskip 1mm% não altere esse espaçamento
        \selectlanguage{brazil}
        {\noindent\it Palavras-chaves:} Colocar palavres chave em português aqui 
        \end{keyword}
        
        \selectlanguage{brazil}
        
        \end{frontmatter}

    \section{Introdução}
    Em sistemas veiculares automotores é necessário controlar as vibrações de sistemas mecânicos com a finalidade de minimizar as vibrações transmitidas ao chassi do veículo e proporcionar maior conforto aos seus utilizadores.
    Em veículos é necessário realizar o controle de vibrações transmitidas pela pista, uma vez que estas causam desconforto aos passageiros e diminuem o contato entre pneu e pista, e diminuem a estabilidade do veículo.
    Existem três tipos de suspensão veicular: suspensão passiva, ativa e semiativa. Esta classificação é dada de acordo com a presença e o tipo de controle utilizado para minimizar as vibrações transmitidas ao chassi.
    Sistemas de controle passivo são os sistemas de suspensão convencionais, são compostos por molas, amortecedores e pneus. O sistema de controle passivo só é capaz de atuar em uma banda de frequência restrita, limitando-se a utilização em sistemas com frequências fora desta banda. Este sistema proporciona uma viagem menos confortável e estável ao carro em comparação aos sistemas de amortecimento ativo e semiativo.
    Um sistema de suspensão ativa é um sistema capaz de atuar em diferentes bandas de frequência, através da utilização de atuadores, sensores e sistemas eletrônicos de controle. A sua desvantagem está na elevada quantidade de energia que os componentes utilizados necessitam, necessitando do uso de uma fonte de energia externa, isto implica em um produto final de custo mais elevado quando comparado com sistemas de suspensão passiva.
    Um sistema de suspensão semiativa também apresenta funcionalidade em diversas bandas de frequência e não possui a obrigatoriedade de uma fonte de tensão externa permanente de grande porte. Outra vantagem deste tipo de sistema é que este, na falta de energia, comporta-se como um sistema passivo, agregando mais confiabilidade e segurança ao veículo. Por fim, um sistema de suspensão semiativa possui custo intermediário entre as demais opções.
    Neste trabalho, focou-se sobre o sistema de suspensão semiativa com atuador Magneto-Reológico (MR). Esta categoria de amortecedores/atuadores são baseados em fluídos magneto-reológicos, tecnologia esta que se encontram disponíveis comercialmente em escala industrial e são utilizados em todo o mundo por diversas montadoras de veículos.
    Este trabalho apresentará a implementação de um controlador semiativo \emph{skyhook} aplicado a um sistema de suspensão veicular com modelo não linear com atuador MR.
    \section{Objetivos}
    Projetar e analisar um observador e um controlador por realimentação no espaço de estados para o modelo linearizado de um sistema de suspensão veicular ativa, não-linear, através do controle da amplitude de deslocamento vertical e dos níveis de aceleração, aos quais os passageiros estão submetidos, empregando para esta finalidade um amortecedor magneto-reológico como atuador. 
    Para alcançar o objetivo principal, os seguintes objetivos específicos foram propostos:
    \begin{itemize}
        \item Realizar a modelagem matemática de um sistema de suspensão não linear de um quarto de veículo com dois graus de liberdade
        \item Realizar a modelagem matemática da dinâmica do amortecedor MR
        \item Realizar a construção do modelo completo de um sistema de suspensão semi-ativa, não linear, de um quarto de veículo com dois graus de liberdade empregando amortecedor MR.
        \item Realizar a linearização do modelo completo proposto
        \item Projetar um observador de estados para a estimação das variáveis de estados necessárias para o funcionamento da estratégia de controle \emph{skyhook}
        \item Projetar um controlador \emph{skyhook} de forma a aumentar o nível de conforto e/ou dirigibilidade do veículo
        \item Realizar simulações temporais com o controlador proposto para o sistema linearizado e original.
    \end{itemize}
    \section{Revisão Bibliográfica}
    \subsection{Modelo matemático do sistema de suspensão não linear para um quarto de veículo }
    O modelo de um quarto de carro consiste em isolar uma quarta parte do veiculo e estudar isoladamente o comportamento do sistema de suspensão para esta seção. Para veículos com peso igualmente distribuído, os resultados são muito próximos ao do modelo completo. Geralmente os modelos para um quarto de carro tem 2 graus de liberdade, sendo estes o deslocamento vertical da massa suspensa e da massa não suspensa. Este modelo pode ser observado na figura \ref{fig:massa_mola_nao_linear_controlavel} a seguir. Este modelo é composto por uma massa suspensa que representa a carroceria do veiculo e uma massa não suspensa que representa o conjunto eixo e roda. Estas massas são conectadas pela mola e pelo amortecedor. O contato do veiculo com a pista de rolamento é realizado através do pneu. O sistema é exitado, ou perturbado, pelas irregularidades da pista.
    \FloatBarrier
    \begin{figure}[htbp]
        \begin{centering}
            \includegraphics[width=5cm]{img/massa_mola_nao_linear_controlavel.png}
            \caption{Modelo de suspensão para um quarto de carro.} 
            \label{fig:massa_mola_nao_linear_controlavel}
        \end{centering}
    \end{figure}
    \FloatBarrier
    Na figura \ref{fig:massa_mola_nao_linear_controlavel} temos que $m_s$ representa a massa suspensa que consiste em um quarto da massa total da carroceria do veiculo em \emph{kg}, $m_u$ representa a massa não suspensa ou a massa do eixo e da roda em \emph{kg}, $b_s$ representa o o coeficiente de amortecimento do amortecedor passivo em \emph{Ns/m}, $k_s$ representa o coeficiente de elasticidade do feixe de molas da suspensão, segundo a lei de hooke, em \emph{N/m}, $k_t$ representa o coeficiente de elasticidade do pneu, segundo a lei de hooke, em \emph{N/m}, $x_r$ representa o deslocamento vertical da pista, onde o sufixo $r$ significa \emph{road}, em \emph{m}, $x_w$ representa o deslocamento vertical da roda, onde o sufixo $w$ significa \emph{wheel}, em \emph{m} e $x_c$ representa o deslocamento vertical da carroceria, onde o sufixo $c$ significa \emph{carr}, em \emph{m}. Na mesma figura o simbolo $F$ representa a atuação de um dispositivo amortecedor com características dinâmicas, seja ele ativo ou semi-ativo, em \emph{N}.
    
    O diagrama de corpo livre do sistema pode ser construído tomando-se como referencia a coordenada da posição do eixo da roda $x_w$ como pode ser observado nas figuras \ref{fig:corpo_livre_ms} e \ref{fig:corpo_livre_mu} a seguir:
    \FloatBarrier
    \begin{figure}[htbp]
        \begin{centering}
            \includegraphics[width=5cm]{img/corpo_livre_ms.png}
            \caption{Diagrama de corpo livre para a massa $m_s$.} 
            \label{fig:corpo_livre_ms}
        \end{centering}
    \end{figure}
    \FloatBarrier
    \begin{figure}[htbp]
        \begin{centering}
            \includegraphics[width=5cm]{img/corpo_livre_mu.png}
            \caption{Diagrama de corpo livre para a massa $m_u$.} 
            \label{fig:corpo_livre_mu}
        \end{centering}
    \end{figure}
    \FloatBarrier
    
    Aplicando a segunda lei de Newton $\sum{F}=m.a$, a cada uma das massas separadamente, o sistema para o modelo de um quarto de carro da figura \ref{fig:massa_mola_nao_linear_controlavel} pode ser representado pela seguinte equação:
    
    \begin{equation} \label{eq:massa_mola_linear}
    \begin{split}
        m_{s} \ddot{x}_{c} =&  b_{s}(\dot{x}_{w}-\dot{x}_{c}) + k_{s}(x_{w}-x_{c}) - F\ \\
        m_{u} \ddot{x}_{w} =& -b_{s}(\dot{x}_{w}-\dot{x}_{c}) - k_{s}(x_{w}-x_{c})  k_{t}(x_{w}-x_{r}) + F
    \end{split}
    \end{equation}
    
    A equação \ref{eq:massa_mola_linear} representa o modelo do sistema na forma linear. Porém a mola $k_s$, o amortecedor $b_s$ e o amortecedor dinâmico ativo podem ser modelados através de modelos lineares ou modelos não lineares. \\
    Uma mola linear obedece a lei de Hooke apresentando uma deformação proporcional ao carregamento. Em uma mola não linear o coeficiente de elasticidade da mola cresce exponencialmente conforme se afasta do ponto de equilíbrio estático.
    Para um modelo de mola não linear, podemos utilizar seguinte expressão para a força da mola $ k_{s}(x_{w}-x_{c})$:
    
    \begin{equation} \label{eq:mola_nao_linear}
        k_{s}(x_{w}-x_{c}) = k^{l}_{s}(x_{w}-x_{c})+k^{nl}_{s}(x_{w}-x_{c})^{3}
    \end{equation}
        
    Na equação \ref{eq:mola_nao_linear} o coeficiente $k^{l}_{s}$ representa o coeficiente de elasticidade do termo da faixa de operação linear e o coeficiente $k^{nl}_{s}$] representa o coeficiente de elasticidade do termo da faixa de operação não linear do feixe de molas em uma situação real.\\
    A não linearidade do amortecedor permite que pequenos movimentos causados pelo perfil da estrada gerem apenas um pequeno impacto na carroceria além de apresentar saturação e histerese. Para um modelo de amortecedor não linear, podemos utilizar a seguinte expressão:
 
    \begin{equation} \label{eq:amortecedor_nao_linear}
        \begin{aligned}
        b_{s}(\dot{x}_{w}-\dot{x}_{c}) =\ \ &b^{l}_{s}(\dot{x}_{w}-\dot{x}_{c}) - b^{y}_{s}\mid\dot{x}_{w}-\dot{x}_{c}\mid \\
        + &b^{nl}_{s}\sqrt{\mid\dot{x}_{w}-\dot{x}_{c}\mid}sgn(\dot{x}_{w}-\dot{x}_{c})  
        \end{aligned}
    \end{equation}
    
    Na equação \ref{eq:amortecedor_nao_linear} o coeficiente $b^{l}_{s}$ representa o coeficiente de amortecimento do termo da faixa de operação linear, o coeficiente $b^{l}_{s}$ representa o coeficiente de amortecimento do termo da faixa de operação não linear e o coeficiente $b^{y}_{s}$ representa a característica de comportamento assimétrico do amortecedor.
    
    Substituindo as equações \ref{eq:mola_nao_linear} e \ref{eq:amortecedor_nao_linear} em \ref{eq:massa_mola_linear} obtemos as seguintes equações diferenciais dinâmicas de segunda ordem que representam a dinâmica de um sistema de suspensão ativa não linear:
    
        \begin{equation} \label{eq:massa_mola_nao_linear}
        \begin{aligned}
         m_{s} \ddot{x}_{c} =\ \ &k^{l}_{s}(x_{w}-x_{c})+k^{nl}_{s}(x_{w}-x_{c})^{3}+b^{l}_{s}(\dot{x}_{w}-\dot{x}_{c})\\
                            -&b^{y}_{s}\mid\dot{x}_{w}-\dot{x}_{c}\mid+b^{nl}_{s}\sqrt{\mid\dot{x}_{w}-\dot{x}_{c}\mid}sgn(\dot{x}_{w}-\dot{x}_{c})\\ 
                            -&F\\
         m_{u} \ddot{x}_{w} =-&k^{l}_{s}(x_{w}-x_{c})-k^{nl}_{s}(x_{w}-x_{c})^{3}-k_{t}(x_{w}-x_{r})\\ 
                            -&b^{l}_{s}(\dot{x}_{w}-\dot{x}_{c})+b^{y}_{s}\mid\dot{x}_{w}-\dot{x}_{c}\mid\\
                            -&b^{nl}_{s}\sqrt{\mid\dot{x}_{w}-\dot{x}_{c}\mid}sgn(\dot{x}_{w}-\dot{x}_{c})+F\\
        \end{aligned}
    \end{equation}
    
    \subsection{Amortecedor Magneto-Reológico}
    Os fluídos magneto-reológicos são um tipo de fluído controlável, sua principal característica é sua habilidade de transitar do estado líquido viscoso para semissólido quando expostos a uma variação de campo magnético em um intervalo de tempo de  milissegundos. Essa característica é demonstrado na figura \ref{fig:fluido_MR} abaixo.
    
    \FloatBarrier
    \begin{figure}[htbp]
        \begin{centering}
            \includegraphics[width=8cm]{img/fluido_MR.png}
            \caption{Etapas do funcionamento do fluído MR quando exposto a um campo magnético} 
            \label{fig:fluido_MR}
        \end{centering}
    \end{figure}
    \FloatBarrier
    
    Pode-se observar na figura \ref{fig:fluido_MR}.1 que, na ausência de um campo magnético, as partículas ferromagnéticas encontradas na estrutura do fluído MR se apresentam todas dispersas e em desordem. Em \ref{fig:fluido_MR}.2 pode-se notar que,ao aplicar um campo magnético, este provoca a magnetização das partículas ferromagnéticas presentes no fluido. Estas partículas adquirem um momento dipolo alinhado ao campo magnético aplicado. Em \ref{fig:fluido_MR}.3 é possível notar que as partículas ferromagnéticas magnetizadas criam cadeias lineares paralelas ao campo aplicado, estes campos paralelos criam uma força coesiva entre as partículas e gera resistência ao movimento do fluído. Finalmente em \ref{fig:fluido_MR}.5 o campo magnético é retirado e observa-se que as partículas retornam ao seu estado inicial sem remanência. \\
    Estas características dos fluídos MR os fazem bastantes interessantes quando aplicados em sistemas mecânicos, devido ao fato destes apresentarem uma arquitetura simples e uma rápida resposta quando fazem interface entre controladores eletrônicos.\\
    Existem basicamente três tipos de Geometria para os amortecedores magneto-reológicos: Amortecedores do tipo monotubo, tipo duplo tubo e tipo haste dupla. Este trabalho se limitará a descrever o tipo mais comum de amortecedor utilizado na industria automotiva que é o amortecedor de tipo monotubo.
    A geometria de amortecedor do tipo monotubo é a mais utilizada entre os amortecedores MR, ela é constituída por quatro elementos principais: um cilindro fechado em uma das extremidades que contém o fluído MR. Um êmbolo, que acomoda a bobina eletromagnética, e se localiza no interior do tubo com capacidade de se deslocar no sentido axial deste. Um acumulador de pressão que contém um gás inerte, tipicamente Nitrogênio, sob pressão de aproximadamente 20 bar e por fim uma haste que faz conexão mecânica com o sistema de suspensão veicular. Existem também outros elementos de conexão e vedação comuns e necessários ao funcionamento do amortecedor, estes porém, não serão comentados neste trabalho. Um amortecedor do tipo monotubo pode ainda possuir duas configurações distintas: Na primeira configuração o êmbolo do amortecedor possui furos passantes que realizam a comunicação do fluido entre as duas câmaras formadas pelos espaços criados no tubo, adiante e a posteriori do êmbolo. A segunda configuração é em corte, onde o fluido se desloca entre as câmaras através de uma folga lateral existente entre o embolo e o tubo. A geometria do tipo monotubo é apresentada adiante na figura \ref{fig:amortecedor_MR_monotubo}:
    
    \FloatBarrier
    \begin{figure}[htbp]
        \begin{centering}
            \includegraphics[width=6cm]{img/amortecedor_MR_monotubo.png}
            \caption{Geometria monotubo com funcionamento em modo válvula a esquerda e modo corte a direita} 
            \label{fig:amortecedor_MR_monotubo}
        \end{centering}
    \end{figure}
    \FloatBarrier
    
    Na figura \ref{fig:curva_amortecedor_MR} a seguir é exibida a curva característica de força-velocidade de um amortecedor dinâmico semi-ativo construído empregando um fluido MR. A força de resposta do amortecedor é exibida como uma função da corrente aplicada às bobinas geradoras de campo e da velocidade de deslocamento do êmbolo:
    
    \FloatBarrier
    \begin{figure}[htbp]
        \begin{centering}
            \includegraphics[width=6cm]{img/curva_amortecedor_MR.png}
            \caption{Família de curvas características da força-velocidade de um amortecedor MR em função da corrente aplicada} 
            \label{fig:curva_amortecedor_MR}
        \end{centering}
    \end{figure}
    \FloatBarrier
    
    A força realizada pelo amortecedor magneto-reológico pode ser descrita através da equação \ref{eq:forca_amortecedor_MR} a seguir:
    
    \begin{equation} \label{eq:forca_amortecedor_MR}
    \begin{aligned}
    \dot{z}=&(\dot{x}_{w}-\dot{x}_{c})-\sigma_{0}.a_{0}.|\dot{x}_{w}-\dot{x}_{c}|.z\\ 
    F=&\sigma_{a}.z +\sigma_{0}.z.\nu+\sigma_{1}.\dot{z}+\sigma_{2}.(\dot{x}_{w}-\dot{x}_{c})+\sigma_{b}.(\dot{x}_{w}-\dot{x}_{c}).\nu
    \end{aligned}
    \end{equation}
 
    Em que:
    
    \begin{itemize}
        \item $F$ : Força aplicada pelo amortecedor MR em [$N$].
        \item $\nu$ : Tensão elétrica aplicada na bobina em [$V$].
        \item $z$ : Variável de estado responsável pela histerese.
        \item $a_{0}$ : Coeficiente do modelo em [${V}/{N}$].
        \item $\sigma_{0}$ : Coeficiente do modelo em [${N}/{m.V}$].
        \item $\sigma_{1}$ : Coeficiente do modelo em [${N.s}/{m}$].
        \item $\sigma_{2}$ : Coeficiente do modelo em [${N.s}/{m}$].
        \item $\sigma_{a}$ : Coeficiente do modelo em [${N}/{m}$].
        \item $\sigma_{b}$ : Coeficiente do modelo em [${N.s}/{m.V}$].
    \end{itemize}
    
    \subsection{Modelo completo de um sistema de suspensão semi-ativa, não linear, de um quarto de veículo, com dois graus de liberdade, empregando amortecedor MR}
    
    Pela aplicação das equações \ref{eq:forca_amortecedor_MR} no sistema descrito em \ref{eq:massa_mola_nao_linear} obtemos as seguintes equações diferenciais dinâmicas de segunda ordem que representam a dinâmica de um sistema de suspensão ativa não linear com atuador dinâmico Magneto-Reológico: 
    
            \begin{equation} \label{eq:massa_mola_nao_linear}
        \begin{aligned}
          m_{s}\ddot{x}_{c}=\ \ &k^{l}_{s}(x_{w}-x_{c})+k^{nl}_{s}(x_{w}-x_{c})^{3}+b^{l}_{s}(\dot{x}_{w}-\dot{x}_{c})\\
                            -&b^{y}_{s}\mid\dot{x}_{w}-\dot{x}_{c}\mid+b^{nl}_{s}\sqrt{\mid\dot{x}_{w}-\dot{x}_{c}\mid}sgn(\dot{x}_{w}-\dot{x}_{c})\\ 
                            -&\sigma_{a}.z-\sigma_{0}.z.\nu+\sigma_{1}.\dot{z}+\sigma_{2}.(\dot{x}_{w}-\dot{x}_{c})+\sigma_{b}.(\dot{x}_{w}-\dot{x}_{c}).\nu\\
          m_{u}\ddot{x}_{w}=-&k^{l}_{s}(x_{w}-x_{c})-k^{nl}_{s}(x_{w}-x_{c})^{3}-k_{t}(x_{w}-x_{r})\\ 
                            -&b^{l}_{s}(\dot{x}_{w}-\dot{x}_{c})+b^{y}_{s}\mid\dot{x}_{w}-\dot{x}_{c}\mid\\
                            -&b^{nl}_{s}\sqrt{\mid\dot{x}_{w}-\dot{x}_{c}\mid}sgn(\dot{x}_{w}-\dot{x}_{c})\\
                            +&\sigma_{a}.z+\sigma_{0}.z.\nu-\sigma_{1}.\dot{z}-\sigma_{2}.(\dot{x}_{w}-\dot{x}_{c})-\sigma_{b}.(\dot{x}_{w}-\dot{x}_{c}).\nu\\
                     \dot{z}=\ \ &(\dot{x}_{w}-\dot{x}_{c})-\sigma_{0}.a_{0}.|\dot{x}_{w}-\dot{x}_{c}|.z\\ 
        \end{aligned}
    \end{equation}
    
    
    \section{Metodologia}
    \section{Resultados}
    \section{Conclusão}
    
    \bibliography{references,manual}
    
    \appendix
    \section{Lista de siglas e abreviações}
    \begin{itemize} 
        \item [\emph{MR}] Magneto Reológico.
        \item [$m_s$] Massa suspensa.
        \item [$m_u$] Massa não suspensa.
        \item [$b_s$] Coeficiente de amortecimento do amortecedor passivo. 
        \item [$k_s$] Coeficiente de elasticidade do feixe de molas da suspensão.
        \item [$k_t$] Coeficiente de elasticidade do pneu.
        \item [$x_r$] Deslocamento vertical da pista.
        \item [$x_w$] Deslocamento vertical da roda.
        \item [$x_c$] Deslocamento vertical da carroceria.
        \item [$F$] Força aplicada pelo amortecedor ativo ou semi-ativo.
        \item [$k^{l}_{s}$] Coeficiente de elasticidade do termo linear no modelo não linear do feixe de molas da suspensão.
        \item [$k^{nl}_{s}$] Coeficiente de elasticidade do termo não linear no modelo não linear do feixe de molas da suspensão.
        \item [$b^{l}_{s}$] Coeficiente de amortecimento do termo da faixa de operação linear do amortecedor.
        \item [$b^{l}_{s}$] Coeficiente de amortecimento do termo da faixa de operação não linear do amortecedor.
        \item [$b^{y}_{s}$] Coeficiente que representa a característica  de comportamento assimétrico do amortecedor.
    \end{itemize}
    
\end{document}